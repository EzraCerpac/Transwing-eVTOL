\chapter{General Summary}
\label{ch:General_Summary}
This chapter will give a broad overview of what the "Transwing eVTOL" project consists of.
In \cref{sec:description-of-entire-system} more context about the project will be given, from which the Mission Need Statement (MNS) and the Project Objective Statement (POS) will flow.
\Cref{sec:customer-requirements} will lastly state the top-level requirements defined by the customer interested in the development and establishment of this technology.

\section{Description of Entire System}
\label{sec:description-of-entire-system}
This project aims to fill a gap in the aviation market, namely Inter-Urban Air Mobility (I-UAM).
UAM is an emerging branch of Air Mobility rapidly gaining traction in both the industry and scientific community \cite{UAM-Overview}.
It entails the use of small, highly automated aircraft for passenger or cargo transportation in urban areas, similar to cars used today.
Vertical Take-Off and Landing (VTOL) vehicles are preferred when constraints are put on the maximum ground footprint of the aircraft.
The currently certified VTOL models for UAM mostly consist of wingless rotorcrafts.
However, wings offer many advantages when it comes to efficient (cruise) flight over longer distances; to achieve I-UAM, and thereby facilitate travel between cities, wings offer a big advantage in terms of performance and design.
To design a low ground footprint VTOL, a transwing option is proposed.
Finally, the vehicle should be electric to reduce emissions, and noise pollution in urban areas and therefore increase product sustainability.
The above considerations result in the following Mission Need Statement (MNS) and Project Objective Statement (POS).

\paragraph{MNS:}“\textit{Achieve Sustainable Inter-Urban Air Mobility with a low ground footprint vehicle.}”
\vspace{-5mm}
\paragraph{POS:}“\textit{Design a low ground footprint, sustainable, urban transwing electric Vertical Take-Off and Landing vehicle within production costs of \num{2000}kEUR, by 10 students in 10 weeks time.}”



\section{Requirements}\label{sec:customer-requirements}
The first set of requirements given can be divided into the following main categories: performance, safety, customer, cost \& sustainability.
Below, the top-level system requirements are listed in the words of the customer. These initial requirements will be further elaborated upon in the first phases of the design. 

% \begin{table}[H]
% \centering
% \caption{Vehicle Requirements Specification}
% \label{tab:vehicle_requirements}
% \begin{tabular}{|l|p{10cm}|}
% \hline
% \textbf{ID} & \textbf{Requirement} \\ \hline
% \multicolumn{2}{|c|}{\textbf{Performance}} \\ \hline
% REQ-001 & Maximum payload weight of 400 kg \\ \hline
% REQ-002 & Indicative maximum size is 8 x 4 x 1 m\textsuperscript{3} on the ground \\ \hline
% REQ-003 & Cruise speed of 200 km/h \\ \hline
% REQ-004 & Compliance with European noise emission regulations \\ \hline
% REQ-005 & Electric power autonomy for 100 km + reserve \\ \hline
% REQ-006 & Lifetime of the vehicle should be greater than 10 years \\ \hline
% \multicolumn{2}{|c|}{\textbf{Safety}} \\ \hline
% REQ-007 & Adoption of an autopilot and proximity sensors \\ \hline
% REQ-008 & Operational at 5 m distance from people and 3 m from any object in densely populated areas \\ \hline
% REQ-009 & Compliance with aero-acoustic regulations for emissions over urban areas, and maximum cabin noise of 60 dBA \\ \hline
% \multicolumn{2}{|c|}{\textbf{Customer Requirements}} \\ \hline
% REQ-010 & Capability for overnight flight \\ \hline
% REQ-011 & Flight in windy conditions, up to 8 Beaufort, with or without rain \\ \hline
% REQ-012 & Redundant propulsive system that can fly with 75\% of the system actively working \\ \hline
% \multicolumn{2}{|c|}{\textbf{Cost and Sustainability}} \\ \hline
% REQ-013 & Final product cost not exceeding 2000 kEUR \\ \hline
% REQ-014 & Main structure should be reusable for at least 10 years and then easily recyclable \\ \hline
% \end{tabular}
% \end{table}


\begin{table}[H]
\centering
\caption{Vehicle Requirements Specification}
\label{tab:vehicle_requirements}
\renewcommand{\arraystretch}{0.85}
\begin{tabularx}{\linewidth}{|l|>{\centering\arraybackslash}X|}
\hline
\textbf{ID} & \textbf{Performance Requirements} \\ \hline
REQ-PERF-001 & Maximum payload weight of 400 kg \\ \hline
REQ-PERF-002 & Indicative maximum size is 8 x 4 x 1 m\(^3\) on the ground \\ \hline
REQ-PERF-003 & Cruise speed of 200 km/h \\ \hline
REQ-PERF-004 & Compliance with European noise emission regulations \\ \hline
REQ-PERF-005 & Electric power autonomy for 100 km + reserve \\ \hline
REQ-PERF-006 & Lifetime of the vehicle should be greater than 10 years \\ \hline
\textbf{ID} & \textbf{Safety Requirements} \\ \hline
REQ-SAF-001 & Adoption of an autopilot and proximity sensors \\ \hline
REQ-SAF-002 & Operational at 5 m distance from people and 3 m from any object in populated areas \\ \hline
REQ-SAF-003 & Maximum cabin noise of 60 dBA \\ \hline
\textbf{ID} & \textbf{Customer Requirements} \\ \hline
REQ-CUST-001 & Capability for overnight flight \\ \hline
REQ-CUST-002 & Flight in windy conditions, up to 8 Beaufort, with or without rain \\ \hline
REQ-CUST-003 & Redundant propulsive system that can fly with 75\% of the system actively working \\ \hline
\textbf{ID} & \textbf{Cost and Sustainability Requirements} \\ \hline
REQ-COST-001 & Final product cost not exceeding 2000 kEUR \\ \hline
REQ-SUST-001 & Main structure should be reusable for at least 10 years and then easily recyclable \\ \hline
\end{tabularx}
\end{table}

