\chapter{Project Guidelines}\label{ch:project_rules}
All team members agreed upon following a set of guidelines to ensure a smooth and efficient workflow during the DSE.
These guidelines are set up to ensure that all team members are on the same page and that the project is executed in a structured and organised manner, and are stated in the following sections.

\paragraph{Contact \& Communication}
\begin{itemize}
    \item The Communication Officer is responsible for formal contact with tutors, coaches and teaching assistants.
    \item Emails are written as a group and checked by the Communication Officer before being sent.
    \item Informal communication within the team will be done via WhatsApp.
\end{itemize}

\paragraph{Schedule}
\begin{itemize}
    \item The group shall be notified as soon as possible when a team member is aware of any absence during the DSE.
    \item When a team member arrives late to a session, he is obliged to bring a cake or something equivalent to the next session.
    \item A team member or work group shall notify the appropriate parties timely when running behind schedule.
    \item Everyone should feel encouraged to ask for help when deemed necessary.
\end{itemize}

\paragraph{Meeting}
\begin{itemize}
    \item During meetings, all team members are expected to give their attention fully, and thus no other work shall be executed simultaneously.
    \item Only English shall be spoken during the meetings and the entire project.
    \item At the start of the working day, a ten-minute stand-up meeting shall take place to:
    \begin{itemize}
        \item discuss the tasks and responsibilities of all individual team members;
        \item set up the whiteboard;
        \item assure that all team members are notified and conscious of the agenda of the meeting with the coaches.
    \end{itemize}
    \item At the end of the working day, a fifteen-minute shutdown meeting shall take place, where the tasks done for the day will be discussed and the timetable will be filled in.
\end{itemize}

\paragraph{Planning}
\begin{itemize}
    \item The Gantt Chart (\cref{sec:gantt-chart}) shall be used for the official planning of the project.
    \item Trello boards shall be used for day-to-day task generation and tracking.
    \item It is each team member’s responsibility to keep their tasks up to date on the Trello board and be informed of the tasks executed by other team members.
\end{itemize}

\paragraph{Documenting}
\begin{itemize}
    \item LaTeX shall be used for the generation of all report deliverables.
    \item All LaTeX projects shall be stored and versioned in GitHub, as well as on the Overleaf platform for the most recent version.
    \item Major versions of the report shall be exported as PDFs and stored in the directory named "Important reports versions" in the shared Microsoft Teams drive.
    \item The following LaTeX guidelines are set up, these shall be enforced by the documents manager and software manager:
    \begin{itemize}
        \item For notes, comments, and tasks within the LaTeX projects, the \verb|/TODO| package shall be utilised.
        Before handing in the report these \verb|/TODO| notes should be removed.
        \item Every chapter, section, figure, table, and equation shall be labelled.
        \item Every figure and table shall have a caption.
        \item Every figure shall be stored in the \verb|figures| directory, in a separate subdirectory for each chapter, and preferably as a PDF file or vector graphic.
        \item Every sentence shall start on a new line and paragraphs shall be divided by an empty line.
        \item For referencing, the \verb|cleveref| package shall be used.
        This entails the use of i.e\@. the \verb|\cref{}| and \verb|\Cref{}| commands.
        \item For numerical values and units the \verb|siunitx| package shall be used.
        This entails the use of i.e\@. the \verb|\qty{}{}| command.
        %    \item For acronyms and abbreviations the \verb|acronym| package shall be used.
        %    This entails the use of i.e\@. the \verb|\ac{}|, \verb|\Ac{}|, and \verb|\acf{}| commands.
    \end{itemize}
\end{itemize}

\paragraph{Language}
\begin{itemize}
    \item All deliverables shall be written in British English.
    \item The use of the passive voice and the first person shall be avoided in the report.
    \item During working hours only English is spoken.
    \item Make correct and frequent use of punctuation in the report.
\end{itemize}

\paragraph{Proofreading}
\begin{itemize}
    \item Clear proofreading guidelines start by adhering strictly to the documenting and language guidelines.
    \item The QAO shall assign all the team members with proofreading responsibilities, which consist of:
    \begin{itemize}
        \item Equal work per member.
        \item Each member being assigned to their expertise and familiarity.
    \end{itemize}
    \item Clear and realistic deadlines are set by the QAO, which consists of multiple rounds of proofreading.
    \item After the first proofreading round, each member proofreads a different chapter/section in a chronological order, and provides it with constructive feedback through comment features and discussion.
    \item Ensure consistency and accuracy by double-checking references and style/punctuation used.
    \item One more proofread round acts as a final review by all members of larger parts, to ensure an error-free and coherent report.
\end{itemize}

\paragraph{Code}
\begin{itemize}
    \item The Python programming language shall be used as much as possible for coding tasks.
    \item All code shall be stored and versioned in the GitHub repository.
    \item The code shall be written in a modular way, with functions, classes, and modules used to separate different parts of the code.
    \item All code shall be properly typed, with type hints used for all functions and classes.
    \item Further, the PEP8 coding standard shall be followed as much as possible.
    \item Each team member shall work in the appropriate branch of the repository.
    \item The Software Manager shall be responsible for merging the branches into the main branch, and overall code quality.
    \item For notes, comments, and tasks within the repository, the \verb|#todo| comment shall be utilised.
    \item For printing messages to the console, the \verb|logging| module shall be used.
    \item Standard functions, for i.e\@. the generation of plots and figures, shall be provided and utilised whenever possible.
\end{itemize}

\paragraph{Computer Aided Modelling (CAD)}
\begin{itemize}
    \item The naming convention (TBD) shall be respected and utilised for all parts and assemblies.
    \item Everything shall and must be iso-constrained.
    \item Whenever possible, parameterised design shall be utilised, since iteration and modification will be a critical part of the design process.
    \item It is encouraged to work in a highly modular manner, with parts and assemblies separated as much as possible.
    \item So-called planes shall be hidden when not in use, to avoid confusion and clutter in the design.
\end{itemize}

\paragraph{Artificial Intelligence}
\begin{itemize}
    \item No generative artificial intelligence shall be used to generate content intended for a deliverable in any way.
    \item Generative artificial intelligence shall not be used to perform or check calculations.
    \item Generative artificial intelligence may be used to generate code lines if, and only if, the generated code is fully understood.
\end{itemize}