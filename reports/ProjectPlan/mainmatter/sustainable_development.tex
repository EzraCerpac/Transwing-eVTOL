\chapter[Sustainability]{Approach to Sustainable Development}\label{ch:sustainability}
Now that the project phases have been carefully planned, and the group organisational structure determined, this chapter focuses on the measures taken within the group to allow sustainable development. 
World institutions such as the United Nations have increasingly highlighted sustainable behaviours in the past decades.
The seventeen sustainable development goals established by the UN in 2016 are a call to the world to promote prosperity, while protecting the planet.
The broadness and multifaceted nature of sustainability is well-reflected in these goals, divided into societal, economic, and environmental actions.
Although socio-economic and environmental aspects are considered at every phase of the design process, the dynamics within the group and the team members' commitment to sustainability are crucial factors in project planning.
Sustainability, characterised by maintaining efficiency and productivity over time, begins with the people involved.
As Vasileva, V. \cite{SustainableTeam} emphasises, a sustainable team - meaning a team that functions cohesively and efficiently - is essential for achieving long-term project success.

The team has thus reached a consensus on fundamental pillars that will serve as the cohesive basis for all actions taken as the project moves forward.
These pillars are listed and defined in \cref{sec:pillars}.
The actions that will be taken to ensure that these pillars are respected are listed in \cref{sec:actions}.
Finally, the major organisational roles to ensure the execution of these actions are listed in \cref{sec:roles}.


\section{Pillars}\label{sec:pillars}
The following major principles have been selected to guide the team dynamics:
\begin{itemize}
    \item \bfemph{Adaptability}: being able to adapt, learn new skills, and face unexpected challenges.
    \item \bfemph{Trust}: creating a safe and pleasurable work environment, with easiness of cooperation.
    \item \bfemph{Transparency}: communicating with the rest of the group about one's work, accessibility of information must be facilitated.
    \item \bfemph{Contribution}: willingness to constantly contribute to the project, help others, and respect the schedule and deadlines.
\end{itemize}
These principles contain overlap, and some listed actions in \cref{sec:actions} may fall within multiple categories.

\section{Guidelines}\label{sec:actions}
Based on the pillars mentioned in \cref{sec:pillars}, the group has set ulterior guidelines to those set in \cref{ch:project_rules} to assure the development of a sustainable team.
The following guidelines have been established and catalogued according to the pillars they fall under.

\paragraph{Adaptability}
\begin{itemize}
    \item Each member shall be willing to take on a task that nobody else was able to perform due to time constraints or other reasons, whether it suits their preferences or not.
    
    \item Each member shall be willing to alter their course of action whenever a new idea arises that the team deems superior.

    \item Each member shall keep a positive attitude towards both their personal work and group efforts, throughout the entire duration of the project, even when faced with hostile circumstances.
\end{itemize}

\paragraph{Trust}
\begin{itemize}
    \item Each member shall behave respectably towards their teammates and external parties.
    
    \item Team bonding activities shall be planned weekly. 

    \item Out-of-topic discussions that contribute towards a positive work environment shall not be discouraged, if it does hinder neither the project's progress nor the team's productivity.
\end{itemize}

\paragraph{Transparency}
\begin{itemize}
    \item Each member shall make information readily accessible to others. 

    \item Team meetings are held at the beginning and end of each project session.
    
    \item Each member or department shall give brief updates before or after a break.

    \item Each member shall communicate the status of their work and any possible delays.
\end{itemize}

\paragraph{Contribution}
\begin{itemize}
    \item Each team member's Personality, Performance, and Potential shall be considered when assigning work packages to improve work efficiency and overall team motivation.

    \item Each member shall take upon themselves an equal share of the project work. 
    
    \item Each member shall be willing to assist other teammates when asked for, or when recognising the possibility to contribute meaningfully.
\end{itemize}



In addition to the actions just described and related to team dynamics, further guidelines have been established which relate to the \bfemph{socio-economical} and \bfemph{environmental} aspects of sustainability:
\begin{itemize}
    \item Team members are encouraged to cycle or walk to the project sessions and meetings.
    \item Team members are encouraged to bring reusable water bottles.
    \item The team shall not waste paper unnecessarily, i.e\@. when the work can be carried out on a whiteboard or electronically.
    \item The team shall not consume the given budget for unnecessary actions.
\end{itemize}


\section{Roles}\label{sec:roles}
Multiple roles established within the organogram in \cref{sec:orgbreakdown} contribute to maintaining a sustainable approach throughout the project.
Namely, the following applies:
\begin{itemize}
    \item Sustainability Officer - The SO guarantees that the guidelines for a sustainable team are followed, and ensures that sustainable choices are made throughout the design process.

    \item Project Manager - The PM assures the effectiveness of the team's work and a fair work share amongst group members.
    The PM can intervene and suggest changes whenever deemed that insufficient progress is made.
    
    \item Communication Officer - The CO mediates between team members in case of misunderstandings and discussions.
    The CO also ensures that the team communicates effectively internally and externally.

    \item Systems Engineer - The SE is crucial to set up concurrent engineering tools.
    These facilitate communication and transparency at all times within the group.
\end{itemize}
